\documentclass[11pt,oneside,english]{amsart}
\usepackage{mathpazo}
\usepackage[scaled=0.92]{helvet}
\usepackage{courier}
\usepackage[T1]{fontenc}
\usepackage[latin9]{inputenc}
\usepackage[a4paper]{geometry}
\geometry{verbose,tmargin=3.2cm,bmargin=3.2cm,lmargin=2.7cm,rmargin=2.7cm}
\usepackage{amsthm}
\usepackage{amssymb}
\usepackage{setspace}
\usepackage[authoryear]{natbib}
\onehalfspacing

\makeatletter
\numberwithin{equation}{section}
\numberwithin{figure}{section}

\newtheorem{theorem}{Theorem}[section]
\theoremstyle{plain}
\newtheorem{acknowledgement}{Acknowledgement}
\newtheorem{axiom}{Axiom}
\newtheorem{proposition}{Proposition}[section]
\newtheorem{conjecture}{Conjecture}[section]
\newtheorem{example}{Example}[section]
\newtheorem{lemma}{Lemma}[section]
\newtheorem{remark}{Remark}[section]
\numberwithin{equation}{section}

\makeatother

\usepackage{babel}
\begin{document}
\title[Multivariate Spectral Subsampling MCMC]{Spectral Subsampling MCMC for Multivariate Time Series}
\author{Robert Kohn, Matias Quiroz and Mattias Villani}
\thanks{
Kohn: \textit{School of Business, University of New South Wales}. 
Quiroz: \textit{University of Technology Sydney}.
Villani: \textit{Dept of Statistics, Stockholm University} and 
\textit{Department of Computer and Information Science, Linkoping University}.
\textit{E-mail: mattias.villani@liu.se}.
}

\begin{abstract}
Whatevs 
\end{abstract}

\maketitle

\section{Introduction}

Spectral subsampling MCMC was proposed in \citet{salomone2019spectral} to accelerate
MCMC for long stationary univariate time series.


\section{Model}
Let $\mathbf{X}_t \in \mathbb{R}^d$ be a $d$-variate zero mean stationary time series with autocovariance
function
\begin{equation}
    \mathbb{C}_{\mathbf{X}}(\tau) = \mathrm{Cov}(\mathbf{X}_t,\mathbf{X}_{t-u})
\end{equation}
The spectral density matrix is
\begin{equation}
    f_{\mathbf{X}}(\omega) = \frac{1}{2\pi}\sum_{\tau=-\infty}^\infty 
        \mathbb{C}_{\mathbf{X}}(\tau)\exp(-i\omega \tau),
\end{equation}
with the diagonal elements is the usual spectral density for each time series and the off-diagonal
elements are cross-spectral densities 
\begin{equation}
    f_{jk}(\omega) = \sum_{\tau=-\infty}^\infty \mathbb{C}_{jk}(\tau)\exp(-i\omega \tau).
\end{equation}   

The Discrete Fourier Transform (DFT) of $\mathbf{X}_t$ is
\begin{equation}
    J_T(\omega) = \sum_{t=0}^{T-1} \mathbf{X}_t \exp(-i\omega t)
\end{equation}
for $\omega \in [-\pi/2,\pi/2]$

The elements of the the DFT $J_T(\omega)$ at the Fourier frequencies for 
$\Omega_T = \{2\pi k/T\}_{k=0}^{[T/2]}$ are asymptotically independent complex multivariate normal 
\citep{brillinger2001time}
\begin{equation}
    J_T(\omega_k) \sim \mathrm{CN}(0,2\pi T f_{\mathbf{X}}(\omega_k)) \text{ as } T\rightarrow \infty.
\end{equation}

The periodogram ordinates $I_T(\omega) = T^{-1}J(\omega)\bar{J}(\omega)^\top$, where $\bar{J}(\omega)^\top$
is the conjugate transpose, are therefore asymptotically 
independent Complex Wishart distributed with one degree of freedom $I_T(\omega) \sim 
CW(1,f_{\mathbf{X}}(\omega))$

The Whittle log-likelihood is therefore
\begin{equation*}
    \ell_\mathcal{W}(\theta) \overset{c}{=} - \sum_{\omega\in\Omega_T}^T \left( \log | f_{\mathbf{X}}(\omega_k)| 
     + \mathrm{tr}\left[f_{\mathbf{X}}(\omega_k)^{-1}I_T(\omega)\right] \right)
\end{equation*}


\section{Experiments}

\section{Conclusions}

\bibliographystyle{apalike}
\bibliography{refMultiSpectralMCMC}


\appendix

\section{Additional results or proofs}
Stuff goes in here.

\end{document}
